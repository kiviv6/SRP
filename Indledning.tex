Da integralregningen blev opfundet i anden halvdel af 1600-tallet med markante bidrag fra blandt andet Newton og Leibniz, var det primært med formålet at bestemme arealer og rumfang.\footnote{\cite{Clausen1993}, s. 129-132.}
Man valgte at definere integration som den omvendte process af differentiation.
Med andre ord sagde man, at integralet for en funktion $f$ var dens stamfunktion $F$, hvor $F'=f$, og med grænser på, var integralet defineret som
\begin{equation}
  \label{eq:stamfunktion}
  \int_{a}^{b} f(x) \,dx =F(b)-F(a).\footnote{Notationen af integralet med grænser blev egentligt først brugt i begyndelsen af 1800-tallet af Fourier, men det ændrer ikke på definitionen af begrebet. Ibid.}
\end{equation}
Dette giver imidlertid anledning til problemer, idet der findes mange funktioner, hvor stamfunktionen ikke eksisterer.
Navnligt er der blandt andet tale om funktioner med bestemte typer af diskontinuiteter.
Eksempelvis kan man vise, at en funktion $f:[0;1] \to \mathbb{R}$ defineret ved 
\[
f(x)= 
\begin{cases}
  2, &\text{ hvis } 0 \leq x <\frac{1}{2}\\
  0, &\text{ hvis } \frac{1}{2} \leq x \leq 1.
\end{cases}
\] 
ikke har nogen stamfunktion.\footnote{Faktummet følger fra den såkaldte mellemværdisætning for afledede funktioner, \cite{Rudin1976}, s. 108.}
Med definitionen af integralet fra 1600-tallet er integralet for funktionen på intervallet $[0;1]$ altså ikke defineret, selvom arealet under grafen for $f$ åbenlyst er $1$. 

Dette motiverer et nyt integralbegreb, der ikke er defineret ud fra differentiation.
I denne artikel introducerer vi da Riemann-integralet, der er en mere generel definition på integralet, hvor blandt andet den ovennævnte funktion er integrabel.
Derudover gælder der, at hvis stamfunktionen $F$ for en funktion $f$ eksisterer, så gælder ligning (\ref{eq:stamfunktion}) stadig.
Sætningen kalder vi for integralregningens hovedsætning.

Siden Riemann-integralet er bygget på nogle bestemte egenskaber for de reele tal, er det netop, hvad vi vil starte med at undersøge i artiklen.
Derefter definerer vi grænseværdi, kontinuitet og uniform kontinuitet stringent med $\varepsilon$-$\delta $-definitioner.
Disse begreber benytter vi senere til at bevise centrale sætninger om Riemann-integralet.
Vi beviser blandt andet, at alle kontinuerte funktioner på et interval $[a;b]$ er Riemann-integrable.
Derudover beviser vi også, at alle begrænsede funktioner med endeligt mange diskontinuiteter er Riemann-integrable.

Dernæst i artiklen præsenteres Simpsons metode, der kan bruges til at approksimere integralet for en kontinuert funktion.
Til sidst findes en diskussion/perspektivering om videreudviklingen af integralbegrebet, hvor enkelte svagheder ved Riemann-integralet undersøges nærmere.


