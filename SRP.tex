\documentclass{article}
\usepackage{graphicx, tikz-cd, float, titlepic, booktabs} % Required for inserting images
\usepackage{pgfplots}
\usepackage{multicol}
\usepackage{makecell}
\pgfplotsset{compat=1.18}
\usepackage{mathrsfs}
\usetikzlibrary{arrows}
\usepackage{amsmath, amssymb, amsthm, amsfonts, siunitx, physics, gensymb}
\AtBeginDocument{\RenewCommandCopy\qty\SI}
\usepackage[version=4]{mhchem}
\usepackage[most,many,breakable]{tcolorbox}
\usepackage{xcolor, fancyhdr, varwidth}
\usepackage[Glenn]{fncychap}
%Options: Sonny, Lenny, Glenn, Conny, Rejne, Bjarne, Bjornstrup
\usepackage{hyperref, cleveref}
\usepackage{icomma, enumitem} %comma as decimal and continue enumerate with [resume]
\usepackage{plimsoll} %use standard state symbol with \stst
\usepackage[danish]{babel}
\renewcommand{\cellalign}{cl}
\renewcommand{\theadalign}{cl}
\renewcommand\theadfont{\bfseries}
\usepackage[round]{natbib} %%Or change 'round' to 'square' for square backers
\usepackage{setspace} %%Enables \doublespacing command for double linespacing
%%%%%%%%%%%%%%%%%%%%%%%%%%%%%%
% SELF MADE COLORS
%%%%%%%%%%%%%%%%%%%%%%%%%%%%%%
\definecolor{myg}{RGB}{56, 140, 70}
\definecolor{myb}{RGB}{45, 111, 177}
\definecolor{myr}{RGB}{199, 68, 64}
\definecolor{mytheorembg}{HTML}{F2F2F9}
\definecolor{mytheoremfr}{HTML}{00007B}
\definecolor{mylenmabg}{HTML}{FFFAF8}
\definecolor{mylenmafr}{HTML}{983b0f}
\definecolor{mypropbg}{HTML}{f2fbfc}
\definecolor{mypropfr}{HTML}{191971}
\definecolor{myexamplebg}{HTML}{F2FBF8}
\definecolor{myexamplefr}{HTML}{88D6D1}
\definecolor{myexampleti}{HTML}{2A7F7F}
\definecolor{mydefinitbg}{HTML}{E5E5FF}
\definecolor{mydefinitfr}{HTML}{3F3FA3}
\definecolor{notesgreen}{RGB}{0,162,0}
\definecolor{myp}{RGB}{197, 92, 212}
\definecolor{mygr}{HTML}{2C3338}
\definecolor{myred}{RGB}{127,0,0}
\definecolor{myyellow}{RGB}{169,121,69}
\definecolor{myexercisebg}{HTML}{F2FBF8}
\definecolor{myexercisefg}{HTML}{88D6D1}
%%%%%%%%%%%%%%%%%%%%%%%%%%%%%%%%%%%%%%%%%%%%%%%%%%%%%%%%%%%%%%%%%%%%%%
% Box environments for theorems and problems
%%%%%%%%%%%%%%%%%%%%%%%%%%%%%%%%%%%%%%%%%%%%%%%%%%%%%%%%%%%%%%%%%%%%%
\setlength{\parindent}{1cm}
%================================
% Question BOX
%================================
\makeatletter
\newtcbtheorem{question}{Opgave}{enhanced,
	breakable,
	colback=white,
	colframe=myb!80!black,
	attach boxed title to top left={yshift*=-\tcboxedtitleheight},
	fonttitle=\bfseries,
	title={#2},
	boxed title size=title,
	boxed title style={%
			sharp corners,
			rounded corners=northwest,
			colback=tcbcolframe,
			boxrule=0pt,
		},
	underlay boxed title={%
			\path[fill=tcbcolframe] (title.south west)--(title.south east)
			to[out=0, in=180] ([xshift=5mm]title.east)--
			(title.center-|frame.east)
			[rounded corners=\kvtcb@arc] |-
			(frame.north) -| cycle;
		},
	#1
}{def}
\makeatother
%================================
% DEFINITION BOX
%================================
\newtheorem{defin}{Definition}[section] % Creates a new counter, number within section

\newtcbtheorem[number within=section, use counter*=defin]{Definition}{Definition}{enhanced,
	before skip=2mm,after skip=2mm, colback=red!5,colframe=red!80!black,boxrule=0.5mm,
	attach boxed title to top left={xshift=1cm,yshift*=1mm-\tcboxedtitleheight}, varwidth boxed title*=-3cm,
	boxed title style={frame code={
					\path[fill=tcbcolback]
					([yshift=-1mm,xshift=-1mm]frame.north west)
					arc[start angle=0,end angle=180,radius=1mm]
					([yshift=-1mm,xshift=1mm]frame.north east)
					arc[start angle=180,end angle=0,radius=1mm];
					\path[left color=tcbcolback!60!black,right color=tcbcolback!60!black,
						middle color=tcbcolback!80!black]
					([xshift=-2mm]frame.north west) -- ([xshift=2mm]frame.north east)
					[rounded corners=1mm]-- ([xshift=1mm,yshift=-1mm]frame.north east)
					-- (frame.south east) -- (frame.south west)
					-- ([xshift=-1mm,yshift=-1mm]frame.north west)
					[sharp corners]-- cycle;
				},interior engine=empty,
		},
	fonttitle=\bfseries,
	title={#2},#1}{def}

\newtcbtheorem[number within=section, use counter*=defin]{definition}{Definition}
{%
	enhanced,
	breakable,
	colback = red!5,
	frame hidden,
	boxrule = 0sp,
	borderline west = {2pt}{0pt}{solid, red!75!black},
	sharp corners,
	detach title,
	before upper = \tcbtitle\par\smallskip,
	coltitle = red!75!black,
	fonttitle = \bfseries\sffamily,
	description font = \mdseries,
	separator sign none,
	segmentation style={solid, red!75!black},
}
{th}

\newtcbtheorem{theo}%
    {Theorem}{}{theorem}
\newtcolorbox{prob}[1]{colback=red!5!white,colframe=red!50!black,fonttitle=\bfseries,title={#1}}
%================================
% NOTE BOX
%================================

\usetikzlibrary{arrows,calc,shadows.blur}
\tcbuselibrary{skins}
\newtcolorbox{note}[1][]{%
	enhanced jigsaw,
	colback=gray!20!white,%
	colframe=gray!80!black,
	size=small,
	boxrule=1pt,
	title=\textbf{Note:},
	halign title=flush center,
	coltitle=black,
	breakable,
	drop shadow=black!50!white,
	attach boxed title to top left={xshift=1cm,yshift=-\tcboxedtitleheight/2,yshifttext=-\tcboxedtitleheight/2},
	minipage boxed title=1.5cm,
	boxed title style={%
			colback=white,
			size=fbox,
			boxrule=1pt,
			boxsep=2pt,
			underlay={%
					\coordinate (dotA) at ($(interior.west) + (-0.5pt,0)$);
					\coordinate (dotB) at ($(interior.east) + (0.5pt,0)$);
					\begin{scope}
						\clip (interior.north west) rectangle ([xshift=3ex]interior.east);
						\filldraw [white, blur shadow={shadow opacity=60, shadow yshift=-.75ex}, rounded corners=2pt] (interior.north west) rectangle (interior.south east);
					\end{scope}
					\begin{scope}[gray!80!black]
						\fill (dotA) circle (2pt);
						\fill (dotB) circle (2pt);
					\end{scope}
				},
		},
	#1,
}
%================================
% EXAMPLE BOX
%================================
\newtcbtheorem[number within=section, use counter from=definition]{example}{Eksempel}
{%
	colback = myexamplebg
	,breakable
	,colframe = myexamplefr
	,coltitle = myexampleti
	,boxrule = 1pt
	,sharp corners
	,detach title
	,before upper=\tcbtitle\par\smallskip
	,fonttitle = \bfseries
	,description font = \mdseries
	,separator sign none
	,description delimiters parenthesis
}
{ex}
%================================
% THEOREM BOX
%================================

\tcbuselibrary{theorems,skins,hooks}
\newtcbtheorem[number within=section, use counter from=definition]{theorem}{Sætning}
{%
	enhanced,
	breakable,
	colback = mytheorembg,
	frame hidden,
	boxrule = 0sp,
	borderline west = {2pt}{0pt}{mytheoremfr},
	sharp corners,
	detach title,
	before upper = \tcbtitle\par\smallskip,
	coltitle = mytheoremfr,
	fonttitle = \bfseries\sffamily,
	description font = \mdseries,
	separator sign none,
	segmentation style={solid, mytheoremfr},
}
{th}

%%%%%%%%%%%%%%%%%%%%%%%%%%%%%%%%%%%%%%%%%%%%%%%%%%%%%%%%%%%%%%%%%
% SELF MADE COMMANDS
%%%%%%%%%%%%%%%%%%%%%%%%%%%%%%
\newcommand{\sol}{\setlength{\parindent}{0cm}\textbf{\textit{Løsning:}}\setlength{\parindent}{1cm}}
\renewcommand\qedsymbol{$\blacksquare$}
%%%%%%%%%%%%%%%%%%%%%%%%%%%%%%%%%
\usepackage[tmargin=2cm,rmargin=1in,lmargin=1in,margin=0.85in,bmargin=2cm,footskip=.2in]{geometry}\pagestyle{fancy}
\lhead{Minrui Kevin Zhou 3.b}

\title{SRP}
\author{Kevin Zhou, 2.b}
\date{\today}

\begin{document}
\include{title.tex}
\begin{abstract}
  Dette er mit resumé.
\end{abstract}

\tableofcontents
\thispagestyle{empty}
\newpage
\section{Indledning}
  \label{sec:Indledning}
%Ordnet legeme def
%\begin{definition}[colbacktitle=red!75!black]{Ordnet legeme}{}
%  Et ordnet legeme er et legeme $\mathbb{F}$ med en delmængde $P \subset \mathbb{F}$, kaldet den positive delmængde med egenskaberne:
%  \begin{itemize}
%    \item Hvis $a \in \mathbb{F}$, så $a \in P$ eller $a=0$ eller $-a \in P$. 
%    \item Hvis $a \in P$, så $-a \not\in P$.
%    \item Hvis $a,\,b \in P$, så $a + b \in P$ og $ab \in P$. 
%  \end{itemize}
%\end{definition}

\section{De reele tal}%
\label{sec:De reele tal}

I denne sektion vil vi først og fremmest undersøge nogle vigtige egenskaber ved $\mathbb{R}$, der er nødvendige for en stringent opbygning af integrationsteorien.
Derefter indfører vi grundlæggende teori om topologi for $\mathbb{R}$, som vi senere skal bruge til at definere kontinuitet samt grænseværdien for en funktion.
\begin{definition}{Overtal, undertal}{}
  \label{def:overtal}
  Antag, at $A \subseteq \mathbb{R}$. 
  Hvis der eksisterer $y \in \mathbb{R}$ sådan at $x \leq y$ for alle $x \in A$, så siges $A$ at være opad begrænset, og $y$ kaldes et overtal for $A$. 

  Hvis der eksisterer $z \in \mathbb{R}$ sådan at $x \geq z$ for alle $x \in A$, så siges $A$ at være nedad begrænset, og $z$ kaldes et undertal for $A$. 
\end{definition}

Imidlertid, kan det også være interessant at kunne sige noget om, om et bestemt overtal er det mindste overtal og et bestemt undertal er det største undertal.

\begin{definition}{Supremum, infimum}{}
  \label{def:sup}
  Antag, at $A \subseteq \mathbb{R}$. Et element $y \in \mathbb{R}$ kaldes supremum eller mindste overtal af $A$ og betegnes $\sup A$, hvis $y$ har følgende egenskaber:
  \begin{itemize}
    \item $y$ er et overtal for $A$.
    \item $y \leq z$ for alle overtal $z$ af $A$. 
  \end{itemize}
\end{definition}
\begin{example}[label=exa:]{Supremum og infimum af mængder}{}
  Betragt mængderne
  \[
  A=\{ a \in \mathbb{R}: a>0 \} \quad \text{og} \quad B=\{ b \in \mathbb{R}:b \leq 1 \}. 
  \] 
\end{example}


\begin{example}[label=exa:supQ]{Ikke-tom opad begrænset delmængde af $\mathbb{Q}$ uden supremum i $\mathbb{Q}$}{}
  
\end{example}

Eksempel \ref{exa:supQ} motiverer den næste sætning, som vi her ikke vil bevise, da beviset er relativt langt og ikke interessant for vores foretagende.
Sætningen er dog en følge af måden, hvorpå $\mathbb{R}$ er konstrueret.\footnote{$\mathbb{R}$ kan konstrueres fra $\mathbb{Q}$ via Dedekind-snit. \cite{Rudin1976}, s. 17-21}

\begin{theorem}{$\mathbb{R}$ har supremum-egenskaben}{}
  Antag, at $A \in \mathbb{R}$, $A \neq \emptyset$ og $A$ er opad begrænset. 
  Så eksisterer $\sup A \in \mathbb{R}$.
\end{theorem}

Denne egenskab ved $\mathbb{R}$ kaldes for supremum-egenskaben. 




\section{Kontinuitet}%
\label{sec:Kontinuitet}

\begin{definition}{Begrænset funktion}{}
  
\end{definition}
\section{Riemann-integralet}%
\label{sec:Riemann-integralet}
I denne sektion vil vi definere Riemann-integralet.\footnote{Sektionen er baseret på \cite{Abbott2002}, s. 183-194}
Før vi gør dette, er vi nødt til at starte med nogle definitioner, vi har brug for.
\begin{definition}{Inddeling}{}
  Antag $a,\;b \in \mathbb{R}$ sådan at $a <b$.
  Ved en inddeling $P$ af $[a;b]$ forstår vi en endelig mængde $\{x_0,\dotsc, x_n\}$, hvor
  \[
  a=x_0<x_1<\cdots<x_n=b
  \] 
  og vi skriver 
  \[
  \Delta x_i=x_i-x_{i-1} \quad (i=1,\ldots ,n)
  \]  
\end{definition}
Med en inddeling kan vi dele intervallet $[a;b]$ op i $n$ delintervaller, hvor det $i$'te delinterval har længden $\Delta x_i$:
\[
[a;b]=[x_0;x_1] \cup [x_1;x_2]\cup \cdots \cup [x _{n-1};x_n]
\] 
Vi vil nu definere over- og undersummen af en funktion med hensyn til en given inddeling.

\begin{definition}{Over- og undersum}{}
  Antag $f:[a;b] \to \mathbb{R}$ er en begrænset funktion, og $P=\{x_0,\ldots, x_n\}$ er en inddeling af $[a;b]$. 
  Lad 
  \[
  M_i=\sup \{ f(x):x \in [x _{x_{i-1}};x_i] \} \quad{\text{og}} \quad m_i=\inf \{ f(x):x \in [x _{x_{i-1}};x_i] \}.
  \] 
  Så er oversummen af $f$ med hensyn til $P$ defineret ved
  \[
  U(f,P)=\sum_{i=1}^{n} M_i \Delta x_i.
  \] 
  Tilsvarende er undersummen af $f$ med hensyn til $P$ defineret ved
  \[
  L(f,P)=\sum_{i=1}^{n} m_i \Delta x_i.
  \] 
\end{definition}
Skriv eksistens

\begin{theorem}{Uligheder med over- og undersummer}{}
  Antag $f:[a;b] \to \mathbb{R}$ er en begrænset funktion og $P_1$, $P_2$ er inddelinger af $[a;b]$ sådan at $P_1 \subseteq P_2$. Så gælder
  \[
  L(f,P_1)\leq L(f,P_2) \leq U(f,P_2) \leq U(f,P_1)
  \] 
\end{theorem}
\begin{proof} 
  j
\end{proof}

\section{Numerisk integration}%
\label{sec:Numerisk integration}


%%%%%%%%%%%%%%%%%%%%%%% BIBLIOGRAPHY %%%%%%%%%%%%%%%%%%%%%%%

\newpage
\singlespacing %%Makes single spaced
\bibliographystyle{apa-good} %%bib style found in bst folder, in bibtex folder, in texmf folder.
\setlength{\bibsep}{5pt} %%Changes spacing between bib entries
\bibliography{Zotero} %%bib database found in bib folder, in bibtex folder
\thispagestyle{empty} %%Removes page numbers
\end{document}
