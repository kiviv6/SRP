I artiklen har vi præsenteret Riemann-integralet, der opklarer mange af de problemer, som definitionen på integralet fra 1600-tallet gav anledning til.
Vi har herunder vist, at alle kontinuerte funktioner på et interval $[a;b]$ er integrable med Riemann-integralet. 
Derudover er alle begrænsede funktioner med endeligt mange diskontinuiteter også integrable.

Integralregningens hovedsætning viser ydermere, at hvis stamfunktionen til en funktion eksisterer (hvilket blandt andet gælder for alle kontinuerte funktioner på intervaller), så kan vi på sædvanlig integrere via stamfunktionsbestemmelse.
Praktisk set er det dog ikke altid lige nemt at finde stamfunktionen til en funktion, og det kan derfor være fordelagtigt at kunne approksimere integralets værdi numerisk.
Vi har her kort præsenteret Simpsons metode, som kan bruges på kontinuerte funktioner.

Trods Riemann-integralets forbedringer ift. 1600-tallet, så findes dog stadig enkelte svagheder ved integralet.
Disse svagheder er senere blevet dækket af Lebesgue-integralet.
