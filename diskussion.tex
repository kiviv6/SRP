Selvom Riemann-integralet løser mange af problemerne, som definitionen på integralet fra 1600-tallet havde, så er Riemann-integralet ikke perfekt, hvilket vi i dette afsnit vil se.\footnote{Afsnittet er baseret på \cite{Axler2020}, s. 9-12, medmindre andet er angivet.}

\subsection{Riemann-integralets svagheder}%
\label{sub:svagheder}
Indtil nu har vi bevist for forskellige typer af funktioner, at de er integrable.
Vi har dog ikke været i stand til at sige, \textit{præcis hvornår} en funktion er Riemann-integrabel.
Hvor nogle funktioner med uendeligt mange diskontinuiteter er integrable, så er andre ikke.
Præcis hvor diskontinuert må en funktion så være, før den er ikke-integrabel?
For at kunne svare på dette spørgsmål, er man nødt til at benytte en generalisering af størrelsen for en mængde, der kaldes for mængdens mål.
Denne målteori blev opfundet af Lebesgue i starten af 1900-tallet, hvor han også indførte et nyt integralbegreb, Lebesgue-integralet.

For at forstå behovet for en ny definition på integralet, vil vi kigge på et eksempel.

\begin{example}[label=exa:arealmening]{Ikke-integrabel funktion, hvor arealet giver mening}{}
  Lad $r_1, r_2, \ldots $ være en følge\footnote{En følge er blot en funktion med de naturlige tal $\mathbb{N}$ som definitionsmængde}, der præcis indeholder hvert rationalt tal i $]0;1[$ én gang. 
  Lad $f_k:[0;1] \to \mathbb{R}$ være defineret ved
  \[
  f_k= 
  \begin{cases}
    \frac{1}{\sqrt{x-r_k} }, &\text{ hvis } x>r_k\\
    0, &\text{ hvis } x \leq r_k,
  \end{cases}
  \] 
  hvor $k \in \mathbb{Z}^+$.
  Lad $f:[0;1] \to \mathbb{R}^+$ være defineret ved
\[
  f(x)= \sum_{k=1}^{\infty} \frac{f_k(x)}{2^k}. 
\] 
  Så har vi, at arealet $A_k$ under grafen for hver $f_k$ må være (se \cref{fig:arealmening})
  \begin{equation*}
  \begin{split}
   A_k &< \int_{0}^{1} \frac{1}{\sqrt{x}} \,dx  \\
  &=\left[2 \sqrt{x} \,\right]_{0}^1 \\
    &=2.
  \end{split}
  \end{equation*}
Arealet under grafen for $f$ må så være
\begin{equation*}
\begin{split}
  A&=\sum_{k=1}^{\infty} \frac{A_k}{2^k}\\
  &<\sum_{k=1}^{\infty} \frac{2}{2^k}\\
  &=2.
\end{split}
\end{equation*}
Man kan vise at rækken konvergerer til $2$.
Vi vil altså gerne have, at arealet under grafen for $f$ skal være mindre end $2$. 
\begin{figure}[H]
\begin{center}
\begin{tikzpicture}[scale=1.6, transform shape]
\begin{axis}[xmin=-0.1, xmax=1.1, ymin=-0.5, ymax=8, axis lines=middle,
  xlabel=$x$,ylabel=$y$,
  xtick={0.2,1},
  xticklabels={$r_k$,1},
  xticklabel style={anchor=north, font=\footnotesize},
  ytick={0},
  yticklabels={$0$},
  every major tick/.append style={thick, major tick length=5pt, black},
  yticklabel style={anchor=east, font=\footnotesize}
  ]
  \clip (axis cs:-0.1,-0.5) rectangle (axis cs:1.2, 8.1);
  \addplot[color=red, domain=0.201:1, samples=100, name path=f, thick]{(x-0.2)^(-1/2)} node[right=5pt, pos=0.9, font=\small] {$f_k=\begin{cases}
    \frac{1}{\sqrt{x-r_k} }, &\text{ hvis } x>r_k\\
    0, &\text{ hvis } x \leq r_k
  \end{cases}
$};
  \draw[color=blue, dashed] (axis cs:0.2,0) -- (axis cs:0.2,8);
  \path [name path=xaxis] (axis cs:0.2,0) -- (axis cs: 1,0);
\addplot[gray!30, opacity=0.9] fill between [of=f and xaxis, soft clip={domain=0.2:1}];
  \draw (axis cs: 0.4,1) node {$A_k<2$};
\end{axis}
\end{tikzpicture}
\end{center}
  \caption{Grafen for $f_k$}%
\label{fig:arealmening}
\end{figure}
Imidlertid har vi for enhver inddeling $P={x_0, \ldots , x_n}$ af $[0;1]$, at alle dens delintervaller $[x_{i-1};x_i]$ (hvor $i=1,\ldots , n$) indeholder et rationalt tal $r_k$, hvilket vil sige, at $f_k$ ikke er begrænset på deltintervallet. 
Man kan så vise, at $f$ af den grund heller ikke er begrænset på alle inddelingens delintervaller.
Altså er Riemann-integralet ikke defineret for $f$ i intervallet $[0;1]$.
\end{example}

Udover det netop viste eksemepel, har vi også vist, at en funktion med alt for mange diskontinuiteter ikke er Riemann-integrabel (se Dirichlets funktion i eksempel \ref{exa:ikke-integrabel}).
Derudover er Riemann-integralet kun defineret for begrænsede funktioner.
Disse er alle sammen gode grunde til den videreudvikling af integrationsteorien, som Lebesgue gav sig til at lave i 1900-tallet.

